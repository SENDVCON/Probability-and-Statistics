\documentclass{article}
\usepackage[utf8]{inputenc}
\usepackage{amsmath}

\title{ECON 520 Problem Set 5}
\author{Ruochen (Frank) Zhou 1849821}
\date{October 2021}

\begin{document}

\maketitle

\section{Poisson Estimation}
An i.i.d. sample is taken from a Poisson distribution, i.e. $f(x;\beta)=\frac{e^{-\beta}\beta^x}{x!}$ for some $\beta>0$ and $x=0,1,2,...$\\\\
(a)  Given that the moment generating function of the Poisson distribution is $M_X(t)=exp(\beta(e^t-1))$ find the mean and variance of $X_i$.\\\\
$M'|_{t=0}=\beta(e^t)exp(\beta(e^t-1)|_{t=0}=\beta$\\\\
$M''|_{t=0}=\beta(e^t)\beta(e^t)exp(\beta(e^t-1)+\beta(e^t)exp(\beta(e^t-1)|_{t=0}=\beta^2+\beta$\\\\
Thus we have\\
$\mu_X=E[X]=\beta$\\
$\sigma^2_X=E[X^2]-E[X]^2=\beta^2+\beta-\beta^2=\beta$\\\\
(b)  Using the method of moments, find an estimator for $\beta$.
\begin{itemize}
    \item Method of Moments\\\\
    Given a sample $x_1,...,X_n$ from an i.i.d. distribution $X_i\simf(x;\theta)$ where the population parameter $\theta$ belongs to the parameter space $\Theta$.  Typically, the oments of the distribution are functions of this paramters $\theta$\\\\
    $E[X^i]=\mu_i(\theta)$ $i=1,...,n$\\\\
    We replace the $j$-th population mean by $\frac{\sum_{i=1}^nX_i^j}{n}$ for $j=1,...,n$ to get \\\\
    $\frac{\sum_{i=1}^nX_i^j}{n}=\mu_i(\theta)$\\\\
    The method of moments estimator $\hat{\theta}$ is defined as the solution to this system of equations.
\end{itemize}
Here, we can replace the first population mean by $\frac{\sum_{i=1}^nX_i}{n}$ to get \\\\
$\hat{\beta}=\frac{\sum_{i=1}^nX_i}{n}$\\\\
(c)  Find the MLE of $\beta$.
\begin{itemize}
    \item Likelihood Function\\\\
    Given a sample $x_1,...,X_n$, the likelihood function of a parameter $\theta$ is defined by\\\\
    $L(\theta|X_1,...,X_n)=f(X_1,...,X_n,\theta)=\prod_{i=1}^nf(X_i,\theta)$
    \item Maximum Likelihood Estimator (MLE)\\\\
    Given a sample $X_1,...,X_n$, the maximum likelihood estimator $\hat{\theta}\in\Theta$ is the parameter value at which $L(\theta|X_1,...,X_n)$ attains a maximum.
\end{itemize}
Here we would like to maximize the likelihood function\\ $L(\theta|X_1,...,X_n)=\prod_{i=1}^nf(X_i,\theta)=\prod_{i=1}^n\frac{e^{-\beta}\beta^{X_i}}{X_i!}$\\\\
As we are concerned only with the maximum, we can perform an order-preserving transformation on our likelihood function to make it easier to work with, and so we may use the log transformation\\
$log(L(\theta|X_1,...,X_n))=log(\prod_{i=1}^n\frac{e^{-\beta}\beta^{X_i}}{X_i!})=\sum_{i=1}^n-\beta+log(\beta^{X_i})-log(X_i!)$\\\\
To maximize the likelihood, we take the first order derivative with respect to $\beta$ and solve the FOC\\
$\frac{\partial}{\partial \beta}log(L(\theta|X_1,...,X_n))=\sum_{i=1}^n-1+\frac{1}{\beta^{X_i}}X_i\beta^{X_i-1}=-n+\frac{\sum_{i=1}^nX_i}{\beta}=0$\\\\
$\hat{\beta}=\frac{\sum_{i=1}^nX_i}{n}$\\\\
(d) Are your ML and MoM estimators unbiased?\\\\
They are indeed unbiased as $E[\frac{\sum_{i=1}^nX_i}{n}]=n\frac{E[X_i]}{n}=\beta$\\\\
(e) Compute the Cramer-Rao Lower Bound.
\begin{itemize}
    \item Cramer-Rao Lower Bound\\\\
    Let $\bf{X}=(X_1,...,X_n)$ be a sample with pdf $f(\bf{x}|\theta)$ and let $\hat{\theta}=\hat{\theta}(\bf{X})$ be an estimator such that $E_\theta[\hat{\theta}]$ is differentiable in $\theta$ interchangeably with expectation.  If $var_\theta[\hat{\theta}]<\infty$ then\\\\
    $var_\theta[\hat{\theta}]\geq\frac{(\frac{d}{d\theta}E_\theta[\hat{\theta}])^2}{E_\theta[(\frac{\partial}{\partial\theta}logf(\bf{X}|\theta))^2]}$\\\\
    Here the denominator is the Fisher information $I(\theta)$
    \item Fisher Information\\\\
    We have two ways to computer the Fisher information $I(\theta)$ which are equivalent\\\\
    i. $nE_\theta[(\frac{\partial}{\partial \theta}logf(X_i|\theta))^2]$\\\\
    ii. $-nE_\theta[\frac{\partial^2}{\partial \theta^2}logf(X_i|\theta)]$
\end{itemize}
Here we have an unbiased estimator and so $E_\theta[\hat{\theta}]=\theta$ and moreover $\frac{d}{d\theta}\theta=1$ and so our CRLB is given by $\frac{1}{I(\theta)}$\\\\
From our computation of the MLE we have\\ 
$log(f(X_i|\theta)=-\beta+log(\beta^{X_i})-log(X_i!)$\\\\
$\frac{\partial}{\partial \theta}logf(X_i|\theta)=-1+\frac{X_i}{\beta}$\\\\
$\frac{\partial^2}{\partial \theta^2}logf(X_i|\theta)=-\frac{X_i}{\beta^2}$\\\\
We can compute the Fisher Information\\
$I(\theta)=-nE_\theta[\frac{\partial^2}{\partial \theta^2}logf(X_i|\theta)]=-n(\frac{-E[X_i]}{\beta^2})=\frac{n}{b}$\\\\
Thus our CRLB is $\frac{1}{I(\theta)}=\frac{\beta}{n}$\\\\
(f) Find the variance of the MLE.\\\\
Since $X_1,...X_n$ are i.i.d. we have\\
$var[\frac{\sum_{i=1}^nX_i}{n}]=\frac{1}{n^2}n(var[X_i])=\frac{\beta}{n}$\\\\
(g) Is the MLE efficient?  Is it consistent?  What about the MoM estimator?\\\\
As the MLE achieves the CRLB, we see that it is efficient.  Moreover, as it is unbiased the variance collapses as $n\rightarrow\infty$, it is consistent.  The same holds for the MoM estimator.  Alternatively, the (weak) LLN shows that our sample mean is a consistent estimator.\\\\
(h) What is the asymptotic distribution of $log(\hat{\beta})$ where $\hat{\beta}$ denotes the MLE?
\begin{itemize}
    \item Central Limit Theorem\\\\
    Let $X_1,...$ be a sequence of iid random variables with\\\\
    $E[X_i]=\mu$ and $var[X_i]=\sigma^2<\infty$\\\\
    Then, $\sqrt{n}(\frac{\bar{X}-\mu}{\sigma})\xrightarrow{d}\mathcal{N}(0,1)$
    \item Delta Method\\\\
    Let $X_1,...X_n$ be a sequence of random variables with\\\\
    $\sqrt{n}(X_n-\mu)\xrightarrow{d}\mathcal{N}(0,\sigma^2)$\\\\
    And let $g$ be a differentiable, non-constant function\\\\
    Then, $\sqrt{n}(g(X_n)-g(\mu))\xrightarrow{d}\mathcal{N}(0,(g'(\mu))^2\sigma^2)$
\end{itemize}
Here we have $X_1,...$ a sequence of iid random variables with $E[X]=\beta$ and finite variance such that $\bar{X}=\hat{\beta}$, Thus, by the CLT we have\\
$\sqrt{n}(\hat{\beta}-\beta)\xrightarrow{d}\mathcal{N}(0,\beta)$\\\\
And further, given the non-constant function $g(x)=log(x)$, by the Delta Method, we have\\
$\sqrt{n}(log(\hat{\beta})-log(\beta))\xrightarrow{d}\mathcal{N}(0,\frac{1}{\beta^2}\beta)$\\\\
$\Rightarrow log(\hat{\beta})\xrightarrow{d}\mathcal{N}(log(\beta),\frac{1}{n\beta})$\\\\
\section{Exponential Estimation}
Let $X_1,X_2,...,X_n$ be i.i.d. with cdf $F_X(x;\lambda)=1-exp(-\lambda x)$ for $0\leq x<\infty$ and $F_X(x;\lambda)=0$ otherwise, and parameter $\lambda>0$.\\\\
(a) Find the pdf for $X_1$.\\\\
$f_X(x;\lambda)=\lambda e^{-\lambda x}$\\\\
(b) Find the MLE of $\lambda$.\\\\
Here we would like to maximize the likelihood function\\ $L(\lambda|X_1,...,X_n)=\prod_{i=1}^nf(X_i,\lambda)=\prod_{i=1}^n\lambda e^{-\lambda x_i}$\\\\
As we are concerned only with the maximum, we can perform an order-preserving transformation on our likelihood function to make it easier to work with, and so we may use the log transformation\\
$log(L(\lambda|X_1,...,X_n))=log(\prod_{i=1}^n\lambda e^{-\lambda x_i})=\sum_{i=1}^nlog(\lambda)-\lambda x_i$\\\\
To maximize the likelihood, we take the first order derivative with respect to $\lambda$ and solve the FOC\\
$\frac{\partial}{\partial \lambda}log(L(\lambda|X_1,...,X_n))=\sum_{i=1}^n\frac{1}{\lambda}-x_i=0$\\\\
$\hat{\lambda}=\frac{n}{\sum_{i=1}^nx_i}$\\\\
(c) Is the MLE unbiased?  Is it consistent?
\begin{itemize}
    \item Jensen's Inequality\\\\
    Let $\phi(x)$ be a (strictly) convex function. Then\\\\
    $\phi(E[x])\leq E[\phi(x)]$\\
    with strict inequality if strictly convex.
\end{itemize}
Consider the strictly convex function $\phi(x)=\frac{1}{x}$\\\\
$E[\hat{\lambda}]=E[\frac{n}{\sum_{i=1}^nx_i}]=E[\phi(\frac{\sum_{i=1}^nx_i}{n})]$\\\\
Here, by Jensen's inequality and given that for an exponential distribution $E[x]=\frac{1}{\lambda}$, we have\\\\
$E[\phi(\frac{\sum_{i=1}^nx_i}{n})]>\phi(E[\frac{\sum_{i=1}^nx_i}{n}])=\lambda$\\\\
Thus, our MLE is biased.  It is consistent as (weak) LLN ensures convergence of the sample mean.\\\\
(d) Does the MLE reach the CRLB?\\\\
First, we compute the Fisher Information\\
$I(\lambda)=-nE_\lambda[\frac{\partial^2}{\partial \lambda^2}logf(X_i|\lambda)]=-nE_\lambda[\frac{-1}{\lambda^2}]=\frac{n}{\lambda^2}$\\\\
$\frac{\partial}{\partial \lambda}E_\lambda[\hat{\lambda}]=\frac{\partial}{\partial \lambda}E_\lambda[\frac{n}{\sum_{i=1}^nx_i}]$\\\\
First, we note that the exponential distribution is in fact a special case of the gamma distribution with shape parameter $k=1$ and scale paramter $\theta=\frac{1}{\lambda}$.  As we have $X_1,...X_n$ are i.i.d. with $X_i\sim gamma(1,\frac{1}{\lambda})$ our MLE actually represents an inverse gamma distribution and so we have  $\frac{1}{\sum_{i=1}^nx_i}\sim gamma^{-1}(n,\frac{1}{\lambda})$.  Note further, the inverse gamma distribution has properties\\
$\mu_x=\frac{\beta}{\alpha-1}$\\
$\sigma^2_x=\frac{\beta^2}{(\alpha-1)^2(\alpha-2)}$ for $\alpha>2$\\\\
Thus we have,\\
$\frac{\partial}{\partial \lambda}E_\lambda[\frac{n}{\sum_{i=1}^nx_i}]=\frac{\partial}{\partial \lambda}n\frac{\frac{1}{\lambda}}{n-1}=\frac{-n}{n-1}\frac{1}{\lambda^2}$\\\\
Then our CRLB is thus\\
$\frac{(\frac{d}{d\lambda}E_\lambda[\hat{\lambda}])^2}{I(\lambda)}=\frac{(\frac{-n}{n-1}\frac{1}{\lambda^2})^2}{\frac{n}{\lambda^2}}=\frac{n}{(n-1)^2}\frac{1}{\lambda^2}$\\\\
The variance of our MLE is given by\\ 
$var[\frac{n}{\sum_{i=1}^nx_i}]=n^2var[\frac{1}{\sum_{i=1}^nx_i}]=n^2\frac{(\frac{1}{\lambda})^2}{(n-1)^2(n-2)}=\frac{n}{n-2}\frac{n}{(n-1)^2}\frac{1}{\lambda^2}$\\\\
And so we see that our MLE fails to reach the CLRB, it is in fact larger by a factor of $\frac{n}{n-2}$.\\\\
(e) Is the MLE efficient?  Is the MLE asymptotically efficient?\\\\
As the MLE failed to reach the CRLB it is not efficient.  However, we see that as $n\rightarrow\infty$ the value $\frac{n}{n-2}\rightarrow1$ and so the MLE is asymptotically efficient.\\\\
(f) Derive the asymptotic distribution of the MLE.\\\\
Here we have $X_1,...$ a sequence of iid random variables with $E[X]=\frac{1}{\lambda}$ and finite variance. Thus, by the CLT we have\\
$\sqrt{n}(\bar{X}-\frac{1}{\lambda})\xrightarrow{d}\mathcal{N}(0,\frac{1}{\lambda^2})$\\\\
And further, given the non-constant function $g(x)=\frac{1}{x}$ by the Delta Method, we have\\
$\sqrt{n}(\frac{1}{\bar{X}}-\lambda)\xrightarrow{d}\mathcal{N}(0,\lambda^2)$\\\\
$\Rightarrow \frac{1}{\bar{X}}\xrightarrow{d}\mathcal{N}(\lambda,\frac{\lambda^2}{n})$\\\\
\section{Binomial Estimation}
Let $X_1,X_2,...,X_n$ be a random sample, where $P(X_i=1)=p$ and $P(X_i=0)=1-p$.\\\\
(a) Consider two estimators of $p$ given by $\hat{p}_1=X_1$ and $\hat{p}_2=\bar{X}$.  Show that both estimators are unbiased.\\\\
$E[\hat{p}_1]=E[X_1]=p$ and $E[\hat{p}_2]=E[\bar{X}]=p$ and thus both estimators are unbiased.\\\\
(b) Intuitively, $\hat{p}_2$ is a better estimator than $\hat{p}_1$ because $\hat{p}_1$ does not contain any information about the parameter contained in trails 2 through $n$.  Verify that speculation by comparing the variances of the two estimators.\\\\
$var[\hat{p}_1]=var[X_1]=p(1-p)$\\
$var[\hat{p}_2]=var[\frac{\sum_{i=1}^nX_i}{n}]=\frac{1}{n^2}var[\sum_{i=1}^nX_i]=\frac{n(p(1-p))}{n^2}=\frac{p(1-p)}{n}$\\\\
Thus we see that $\hat{p}_2$ indeed has lower variance.  
\end{document}
